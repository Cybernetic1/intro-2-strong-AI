\chapter{Natural language}
\begin{flushright}
\emph{The fish trap exists because of the fish; once you've gotten the fish,\\
you can forget the trap. The rabbit snare exists because of the rabbit;\\
once you've gotten the rabbit, you can forget the snare.\\
Language exists because of meaning; once you've gotten the meaning,\\
you can forget language.}\\ --- Zhuangzi (4th Century BC)
\end{flushright}
\minitoc

\section{Natural language is not essential to AGI}

Needless to say, fully solving the natural language problem is AGI-complete.  This however is not a show-stopper.  Our ultimate goal is to reach the \textbf{RSI point} (\S\ref{sec:RSI}) with as little effort as possible.  Which means we only need the most basic system that can automatically write programs for us.  Such a system only needs to accept commands in a \textit{restricted} subset of English.  Thus, for all our purposes an NL module that can parse restricted English would suffice.

\section{Unification-based grammars}

The family of unification-based grammars includes LFG (Lexical Functional Grammar), HPSG (Head-Driven Phrase Structure Grammar), and PATR grammar.  The unification algorithm used in unification-based grammar is the same as the unification algorithm used in logic.  This is further evidence that the brain employs first-order symbolic processing.

\section{Cognitive linguistics}

Currently we are using the ``formal semantics'' approach.

\{ TO-DO:  How may cognitive linguistics affect the design of the natural language subsystem? \}

\section{Abduction as interpretation}
\label{sec:AbductionAsInterpretation}

\subsection{A detailed example}

The general sequence is:\\
\hspace*{1cm} tokenization $\rightarrow$ POS-tagging $\rightarrow$ syntax parsing $\rightarrow$ semantic parsing\\
which should be familiar to everyone with experience in NL processing.

I will explain the details using a simple example:\\
\hspace*{1cm} ``I love Mary''

The crucial thing is that we represent everything in a logic-based framework. First we represent the sentence as "raw data" (ignoring tenses to simplify matters):\\
\hspace*{1cm} $\mbox{sentence}(e_0)$\\
\hspace*{1cm} $\mbox{lexeme-I}(e_1)$\\
\hspace*{1cm} $\mbox{lexeme-love}(e_2)$\\
\hspace*{1cm} $\mbox{lexeme-Mary}(e_3)$\\
\hspace*{1cm} $\mbox{begins-with}(e_0, e_1)$\\
\hspace*{1cm} $\mbox{follows}(e_2, e_1)$\\
\hspace*{1cm} $\mbox{follows}(e_3, e_2)$\\
where:\\
\hspace*{1cm} the $e_i$'s are \textbf{entities} (logical constants).\\
\hspace*{1cm} the entities $e_1$, $e_2$, and $e_3$ are words.\\
\hspace*{1cm} follows() means a word follows another word in a sentence.

Up to now, all we have is a sentence as raw text (without meanings).  The next step is to recognize parts of speech (nouns, verbs, adjectives, etc). We can use logical rules to do this.

An example logical rule is:\\
\hspace*{1cm} $\mbox{lexeme-Mary}(X) \rightarrow \exists e \; \mbox{parse-as}(e, X) \wedge \mbox{noun}(e)$\\
which simply means that the word "Mary" is a noun. X is a variable (implicitly universally quantified). $e$ is a new entity, which instantiates to $e_4$ when the rule is applied.

We can also use logical rules to parse syntax. We can perform ``VP $\leftarrow verb + noun$'' with this rule\footnote{We need this special rule:\\
\hspace*{1cm} $\mbox{follows}(X_1, X_2) \wedge \mbox{parse-as}(X_1, X_3) \wedge \mbox{parse-as}(X_2, X_4) \rightarrow \mbox{follows2}(X_3, X_4)$ }:\\
\hspace*{1cm} $\mbox{verb}(X_1) \wedge \mbox{noun}(X_2) \wedge \mbox{follows2}(X_2, X_1) \rightarrow \exists e \; \mbox{parse-as}(e, X_1, X_2) \wedge vp(e)$\\
which creates a new entity $e_5$ which is a VP.

Assume that eventually we have a parse of the sentence S, $e_6$. Notice that up to now, it's all syntax parsing.

Next we perform semantic parsing.  The key is to generate partial meanings for phrases, such as the verb phrase ``loves Mary''.

\textbf{Lambda operator.}  In formal semantics, it is customary to use the $\lambda$ operator to represent the meaning of phrases.  The reason is that first-order logic do not have the expressive power to represent such phrases.  For example, the VP ``loves Mary'' denotes ``somebody's loving Mary'', which may be represented as $love(\_ \,,mary)$; but that is not a well-formed formula in FOL.  Instead we can represent it using the $\lambda$ expression $\lambda x \; love(x,mary)$.  This is no longer FOL, but is a so-called \textit{quasi-logical} form.  I propose to use an alternative method which is within FOL.  It employs the composition functor.

\textbf{Composition functor.}  (See also \S\ref{sec:CompositionFunctor})  Under this method, all phrases are represented by compositions.  For example, the VP ``loves Mary'' can be represented by $comp(loves,mary)$, or in short-hand $loves * mary$.  This composite is a \textit{first-order object}, which cannot be further reduced, but when we compose it with $john$, we get $john * (loves * mary)$ which is equivalent to the first-order statement $loves(john,mary)$.  So we need special inference rules to convert such composites into statements.

I think this method is superior to $\lambda$ because the meanings of phrases can be represented by first-order objects.  It also makes semantic parsing very intuitive.

\{ TO-DO:  Explain semantic parsing with examples. \}

%A semantic rule is similar to a syntactic one:\\
%\hspace*{1cm} $\mbox{lexeme-love}(X) \rightarrow \exists e \; \mbox{parse-as}(e, X) \wedge \mbox{verb}(e) \wedge \mbox{means}(e, \mbox{concept-love})$\\
%
%\hspace*{1cm} $\mbox{verb}(X_1) \wedge \mbox{np}(X_2) \wedge \mbox{follows2}(X_2, X_1) \wedge \mbox{means}(X_1, Y_1) \wedge \mbox{means}(X_2, Y_2)$\\
%\hspace*{1cm} $\rightarrow \exists e \; \mbox{parse-as}(e, X_1, X_2) \wedge vp(e) \wedge \mbox{means}(e, comp(Y_1,Y_2))$\\
%
%For example, we can augment the syntactic rule\\
%\hspace*{1cm} $\mbox{VP} \leftarrow \mbox{verb}, \mbox{NP} $\\
%with\\
%\hspace*{1cm} $\mbox{VP}(\mbox{comp}(Sem_1,Sem_2)) \leftarrow \mbox{verb}(Sem_1), \mbox{NP}(Sem_2)$\\
%where the $Sem$'s are variables containing the ``semantics'' of phrases.\\
%\hspace*{1cm} $\mbox{S} \leftarrow \mbox{NP}, \mbox{VP} $\\
%\hspace*{1cm} $\mbox{S}(Predicate) \leftarrow \mbox{NP}(Subject), \mbox{VP}(Subject * Predicate) $\\
%\hspace*{1cm} $\mbox{VP}(X) \rightarrow \mbox{com}(\mbox{loves},\mbox{mary})$\\
%which creates a new entity $e_7$ with the meaning of "love".\\
%\hspace*{1cm} $\mbox{lexeme-love}(X_1) \rightarrow \exists e \; \mbox{concept}(e) \wedge \mbox{means}(e, \mbox{concept-love})$\\
%
%The final step of semantic parsing uses a special trick,  ``de-reification'' (\S\ref{sec:reification}):\\
%\hspace*{1cm} (assume that $e_{8}, e_{9}$ denote the persons John and Mary)\\
%\hspace*{1cm} $\mbox{de-reify}(\mbox{concept-love}, e_{8}, e_{9})$\\
%which generates the logical statement:\\
%\hspace*{1cm} $\mbox{loves}_1(e_{8},e_{9})$\\
%
%Finally we have:\\
%\hspace*{1cm} $\mbox{means}(e_0, \mbox{loves}_1)$

\section{Universal logical form}
\label{sec:UniversalLogicalForm}

An immediate question is how to translate ``nearly every text sentence under the sun'' into logical form.  To which we employ several rules to quickly reduce the problem space:

1.  Every noun (except special nouns such as pronouns) corresponds to a concept which is represented by a predicate.  For example, the word:chair corresponds to the concept:chair and is represented by the predicate \code{concept-chair()}.

%2.  Every verb

%3. 

\section{Some example English sentences}
\label{sec:English-examples}

